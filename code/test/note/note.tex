\documentclass{report}
\begin{document}
    \section{Rappresentazione Modelli Stringhe}
        La prima cosa da fare quando lavoriamo con lambeq è processare la stringa
        di input e convertirla in uno string diagram.
        \newline I modelli che possono essere utilizzati per la modellazione delle strighe sono:
            \begin{itemize}
                \item   \textbf{Syntax-based model: DisCoCat}
                        \newline Per ottenere un risultato simile alla notazione DisCoCat, possiamo usare il parser di default BobcatParser
                                 Fornisce una modello syntax-based ovvero completamente basati sulla sintassi della frase e dei collegamenti 
                                 tra le componenti della stringa.
                \item   \textbf{Bag-of-Words model: spiders reader}
                        \newline Modello di tipo bag-of-word, vuol dire che non tiene conto delle connessioni e le relazioni tra i termini della stringa
                                 ma tratta la stringa stessa come un unico insieme di parole da analizzare separatemente.
                \item   \textbf{Word-Sequence model: cups and stairs readers}
                        \newline Questi modelli tengono conto dell'ordine in cui le parole vengono presentate ma ignorano totalemente la sintassi e le
                                 relazioni tra esse.    
                \item   \textbf{Tree-reader}
                        \newline Le stringhe vengono divise in monoidi e collegate in una struttura ad albero in base alle relazioni e alla sintassi che
                                 mano mano viene identificata. 
            \end{itemize}
            
        \leavevmode\newline
        Nel contesto da realizzare, ovvero il riconoscimento e la classificazione dei requisiti di security, credo sia necessario utilizzare un approccio
        che utilizza i modelli che tengono conto della sintassi (anche in parte). 
        \newline Questo perchè se usassimo un modello di tipo bag of words non possiamo identificare realmente frasi sconnesse con parole messe a caso da 
        periodi finiti e ben strutturati. 
        \newline Inoltre bisogna sfruttare la sintassi per capire meglio il corpo di ciascuna frase, anche perchè si potrebbe pensare di estrarre dei 
        concetti e non basarci solamente sulle parole in input per classificare.
        
        \leavevmode\newline
        (esempio: una frase che parla del napoli potrebbe parlare di calcio o di sport)
        (esempio: una frase potrebbe implicitamente dire il suo contesto di classificazione).

        \leavevmode\newline
        Si potrebbe pensare quindi di approcciare per una modellazione basata su DisCoCat oppure, se risulta essere più lightweigth dal punto di vista computazionale e se
        ottiene a grandi linee gli stessi risultati, il modello basato su word-sequence, in modo da non per forza analizzare tutta la frase per estrarne la categoria di appartenenza.
\end{document}